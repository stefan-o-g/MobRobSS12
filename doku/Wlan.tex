\subsection{WLAN - so funktioniert es}
\label{wlan}

\subsubsection{Auf dem Bot}
TODO: welche funktionen werden genutzt, wie sieht ein paket bzw die c-struktur aus, blablabla...

\subsubsection{Auf dem PC}
TODO: einfach ins wlan verbinden und udp paket verschicken blablabla....

\subsection{Zusatzaufgabe}

\subsubsection{ctremote.py - Ein Skript zur Fernsteuerung des Bots}
\label{ctremote}
Das Skript ist ein interaktives Programm, welches dem Nutzer ermöglicht Befehle einzugeben, die Aktionen auf Seiten des Bots auslösen. Dazu muss auf dem Bot das Programm laufen, welches in Abschnitt \ref{ctremote_bot} beschrieben ist.

Nachfolgend die bisher implementierten Befehle:
\begin{itemize}
	\item subcmd <subcommand>\\
	Sendet den angegebenen Subcommand <subcommand> an den Bot. Es sind bisher folgende Subcommands implementiert:
	\begin{itemize}
		\item 1 oder stand\\
		Lässt den Bot anhalten.
   		\item 2 oder motte1\\
   		Lässt den Bot die erste Implementierung von "'Motte"' ausführen.
   		\item 3 oder kakerlake1\\
   		Lässt den Bot die erste Implementierung von "'Kakerlake"' ausführen.
   		\item 4 oder motte2\\
   		Lässt den Bot die zweite Implementierung von "'Motte"' ausführen.
   		\item 5 oder kakerlake2\\
   		Lässt den Bot die zweite Implementierung von "'Kakerlake"' ausführen.
   		\item 6 oder acht\\
   		Lässt den Bot eine Acht fahren.
   		\item 7 oder linie\\
   		Lässt den Bot eine Linie entlang fahren.
	\end{itemize}
	Beispiel: \textit{subcmd motte1}
	
	Beispiel 2 : \textit{subcmd 6}
	\item move\\
	Nach Eingabe dieses Befehls ist der Bot über die Tastatur fernsteuerbar. Die Tastenbelegung dafür ist wie folgt:
	\begin{itemize}
		\item W - Vorwärts fahren.
		\item S - Rückwärts fahren.
		\item A - Nach links drehen.
		\item D - Nach rechts drehen.
		\item E - Anhalten.
		\item Q - Beendet den move-Befehl und kehrt zur Befehlseingabe zurück.
	\end{itemize}
	\item get <what>\\
	Zeigt einem den aktuellen Wert von <what> an. Dabei kann <what> momentan \textit{botip} (also die IP des zu steuernden Bots) oder \textit{port} (also der UDP-Port, über den kommuniziert wird) sein.
	
	Beispiel: \textit{get botip}
	\item set <what> <value>\\
	Setzt den Wert von <what> auf <value>. Hierbei kann <what> wieder \textit{botip} oder \textit{port} sein.
	
	Beispiel: \textit{set botip 192.168.0.9}
	\item help\\
	Gibt einen kurze Übersicht aller Befehle mit einer kurzen Beschreibung aus.
	\item help <cmd>\\
	Gibt einen detaillierten Hilfetext zum angegebenen Befehl <cmd> aus.
	
	Beispiel: \textit{help subcmd}
	\item exit\\
	Beendet das Skript.
	\item quit\\
	Beendet das Skript.
\end{itemize}
Die Tastenkombination \textit{STRG+C} wird von dem Skript abgefangen und sendet dem Bot sofort den Befehl zum Anhalten, beendet das Skript jedoch nicht. Um das Skript zu beenden muss einer der oben beschrieben Befehle \textit{quit} oder \textit{exit} verwendet werden.


\subsubsection{Programm auf dem Bot}
\label{ctremote_bot}

