\subsection{Installation}
Für die Entwicklung wurde ein aktuelles Ubuntu Linux verwendet. In diesem Abschnitt ist beschrieben was für vorbereitende Schritte auf dem System durcheführt werden müssen um entwickeln und den ct-Bot flashen zu können.


\subsubsection{System vorbereiten}
Zunächst müssen einige Softwarepakete nachinstalliert werden:
\begin{itemize}
\item eclipse-cdt\\
Die Eclipse-Variante zur C/C++-Entwicklung. (Der ct-Bot wird in C programmiert.)
\item binutils-avr, gcc-avr, avr-libc\\
Werden zum Kompilieren für den Microcontrollers des ct-Bot benötigt. (Cross-Compiler)
\item avrdude\\
Wird zum Flashen des Microcontrollers des ct-Bot benötigt.
\item subversion\\
Wird zum Holen des aktuellen Quellcodes für den ct-Bot aus dem heise-Repository benötigt.
\end{itemize}
Der konkrete Befehl um die Pakete unter Ubuntu zu installieren sieht folgendermaßen aus:
\begin{lstlisting}
	sudo apt-get install eclipse-cdt binutils-avr gcc-avr \
	avr-libc ^avrdude subversion
\end{lstlisting}


\subsubsection{Quellcode holen}
Für den Quellcode erstellen wir zunächst ein Verzeichnis \textit{ctbot} und wechseln hinein:
\begin{lstlisting}
	mkdir ctbot && cd ctbot
\end{lstlisting}
Nun holen wir uns den aktuellen stable-Code vom heise-Repository:
\begin{lstlisting}
	svn checkout https://www.heise.de:444/svn/ctbot/stable
\end{lstlisting}
Der aktuelle Quellcode des ct-Bot befindet sich nun also unter \textit{\~{}/ctbot/stable} und muss im nächsten Schritt nur noch in Eclipse eingebunden werden.


\subsubsection{Eclipse einrichten}
Zunächst müssen wir den Quellcode in Eclipse einbinden:
\begin{itemize}
\item Dazu wählen wir zunächst im Menü \textit{File} den Unterpunkt \textit{Import}.
\item Dort wählen wir \textit{General -> Existing Projects into Workspace} und bestätigen mit \textit{Next >}.
\item Unter der Auswahl \textit{Select root directory} geben wir entweder direkt das Quellcodeverzeichnis an (\textit{\~{}/ctbot/stable/ct-Bot}) oder wählen das Verzeichnis über \textit{Browse...}.
\end{itemize}
Nun ist der Quellcode als Projekt in Eclipse eingebunden. Um für den ct-Bot zu kompilieren muss jedoch noch die Build-Configuration angepasst werden:
\begin{itemize}
\item Auf der linken Seite wählen wir zunächst das Projekt \textit{ct-Bot} aus.
\item Im Menü \textit{Project} wählen wir \textit{Properties} und dort \textit{C/C++-Build}.
\item Dort gehen wir auf \textit{Manage Configurations...} und wählen die Konfiguration \textit{Debug-MCU-m32}. Mit dieser Konfiguration wird der zuvor installierte Cross-Compiler genutzt um eine hex-Datei zum Flashen des ct-Bot zu erstellen.
\item Wir bestätigen die Auswahl der Konfiguration mit \textit{Set active} und verlassen die Einstellungen mit \textit{Ok}, \textit{Apply} und nochmals \textit{Ok}.
\end{itemize}
Nachdem nun auch die passende Konfiguration gewählt wurde lässt sich das Projekt nun auch kompilieren. Jedoch kommt es zu einigen Warnmeldungen.
Um diese Warnungen loszuwerden öffnen wir die Datei \textit{ct-Bot.h} (\textit{include/ct-Bot.h}) und suchen die Zeile
\begin{lstlisting}
//#define SPEED_CONTROL_AVAILABLE
\end{lstlisting}
und entfernen die Kommentarzeichen \textit{//}.
Nun lässt sich das Projekt ohne Warnungen kompilieren und die  eigentliche Entwicklung kann beginnen.

\subsection{Inbetriebnahme}
\subsubsection{AVR ISP mkII und avrdude}
Der \textit{AVR ISP mkII} ist der Programmer, mit dem wir die den ct-Bot flashen können. Also Tool dazu verwenden wir \textit{avrdude}, der den \textit{AVR ISP mkII} ansprechen kann.

Nach dem Einstecken des Programmers können wir mit dem Befehl \textit{lsusb} prüfen, ob er korrekt vom System erkannt wird. Die Ausgabe sollte folgenden Text enthalten:
\begin{lstlisting}
Atmel Corp. AVR ISP mkII
\end{lstlisting}

Um unsere hex-Datei nun auf den ct-Bot zu bekommen, müssen wir \textit{avrdude} mit entsprechenden Parametern aufrufen:
\begin{itemize}
\item \textit{-c avrispmkII} legt den \textit{AVR ISP mkII} als Programmer fest.
\item \textit{-P usb} gibt USB als connection port an.
\item \textit{-p m32} legt m32 (ATmega32) als AVR device fest.
\item \textit{-U flash:w:<pfad>:i} legt die Aktion fest:
	\begin{itemize}
	\item \textit{flash} gibt an, dass geflasht werden soll.
	\item \textit{w} (write) gibt an, dass geschrieben werden soll. (Von der Datei in den Flash-Speicher, \textit{r} (read) würde bedeuten vom Flash in die Datei.)
	\item \textit{<pfad>} gibt den Pfad zu der Datei an.
	\item \textit{i} gibt (optional) das Format der Datei an. (Hier \textit{i} für \textit{Intel Hex}.)
	\end{itemize}
\end{itemize}
Zu beachten ist, dass \textit{avrude} als root oder über \textit{sudo} aufgerufen werden muss.
Der konkrete Aufruf würde also folgendermaßen aussehen:
\begin{lstlisting}
sudo avrdude -c avrispmkII -P usb -p m32 -U \
flash:w:"~/ctbot/stable/ct-Bot/Debug-MCU-m32/ct-Bot.hex":i
\end{lstlisting}
