\subsection{Installation}
Für die Entwicklung wurde ein aktuelles Ubuntu Linux verwendet. In diesem Abschnitt ist beschrieben was für vorbereitende Schritte auf dem System durcheführt werden müssen um entwickeln und den ct-Bot flashen zu können.


\subsubsection{System vorbereiten}
Zunächst müssen einige Softwarepakete nachinstalliert werden:
\begin{itemize}
\item eclipse-cdt\\
Die Eclipse-Variante zur C/C++-Entwicklung. (Der ct-Bot wird in C programmiert.)
\item binutils-avr, gcc-avr, avr-libc\\
Werden zum Kompilieren für den Microcontrollers des ct-Bot benötigt. (Cross-Compiler)
\item avrdude\\
Wird zum Flashen des Microcontrollers des ct-Bot benötigt.
\item subversion\\
Wird zum Holen des aktuellen Quellcodes für den ct-Bot aus dem heise-Repository benötigt.
\end{itemize}
Der konkrete Befehl um die Pakete unter Ubuntu zu installieren sieht folgendermaßen aus:
\begin{lstlisting}
	sudo apt-get install eclipse-cdt binutils-avr gcc-avr \
	avr-libc ^avrdude subversion
\end{lstlisting}


\subsubsection{Quellcode holen}
Für den Quellcode erstellen wir zunächst ein Verzeichnis \textit{ctbot} und wechseln hinein:
\begin{lstlisting}
	mkdir ctbot && cd ctbot
\end{lstlisting}
Nun holen wir uns den aktuellen stable-Code vom heise-Repository:
\begin{lstlisting}
	svn checkout https://www.heise.de:444/svn/ctbot/stable
\end{lstlisting}
Der aktuelle Quellcode des ct-Bot befindet sich nun also unter \textit{\~{}/ctbot/stable} und muss im nächsten Schritt nur noch in Eclipse eingebunden werden.


\subsubsection{Eclipse einrichten}
TODO:
\begin{lstlisting}
#source im eclipse einbinden:
File -> Import
General -> Existing Projects into Workspace
Browse...
~/ctbot/stable/ct-Bot
#evtl nochmal mit ct-Sim wenn gewollt


ct-Bot Projekt wählen #evtl. selbes mit ct-Sim
Project->Properties
C/C++-Build
Manage Configurations...
Debug-MCU-m32
Set active
Ok
Apply
Ok

#SPEED_CONTROL_AVAILABLE aktivieren für Kompilieren ohne Warnungen
in der ct-Bot.h den Kommentar vor
    #define SPEED_CONTROL_AVAILABLE
entfernen...
\end{lstlisting}

\subsection{Inbetriebnahme}
\subsubsection{AVR ISP mkII und avrdude}
TODO:
\begin{lstlisting}
#in betrieb nehmen
Programmer einstecken
lsusb zeigt "Atmel Corp. AVR ISP mkII

#flashen mit avrdude
avrdude
    -c avrispmkII -> legt den Prgrammer fest
    -P usb -> usb als connection port
    -p m32 -> legt das AVR device fest: m32
    -U flash:w:"<file>":i -> flash gibt an dass der Flash-Speicher verwendet werden soll
                          -> w gibt an dass geschrieben werden soll (file -> flash)
                          -> "<file>" is der Pfad zur Datei die ins Flash 
                             geschrieben werden soll (hier ct-Bot.hex)
                          -> i gibt das Format der Datei an, hier i für Intel Hex
sudo!!!

konkret:
 sudo avrdude -c avrispmkII -P usb -p m32 -U flash:w:"~/ctbot/stable/ct-Bot/Debug-MCU-m32/ct-Bot.hex":i

\end{lstlisting}