\subsection{Weg Aufzeichnen}
\label{weg-aufzeichnen}

\subsubsection{Ansatz Verhaltensframework}
Die modifizerte Version der Motte, die sich den gefahrenen Weg merken und diesen
wieder zurückfahren können soll, wurde über das Motte/Kakerlake Verhalten und
vorhandene Verhalten realisiert.
Dazu wurde zuerst das Verhalten \verb+BEHAVIOUR_DRIVE_STACK_AVAILABLE+ zugeschalten \\
(in \verb+ct-Bot/include/bot-logic/available\_behaviours.h+ und anschlißend
auf aktiv gesetzt (\verb+bot_save_waypos_behaviour+ in
\verb+ct-Bot/bot-logic/bot-logic.c+). Dieses Verhalten zeichnet im Folgenden
alle Informationen auf, die für das wiederfinden der relevanten Positonen nötig sind.
Nach einer über die Konstante \verb+MAX_WAYPOINTS+ festgelegten Anzahl Wegpunkten,
wird das \verb+drive_stack()+ Verhlaten aufgerufen das die Punkte wieder anfährt.
Wenn der Roboter an der Ausgangsposition angekommen ist, beginnt er wieder von vorne mit
der Wegaufzeichnung und der Lichtquellensuche. Die eigentliche Änderung im
Motte/Kakerlake Verhalten liegt also nur darin, nach jeder Aktion den Waypoint-Counter
hoch zu zählen und nach der definierten Anzahl Waypoints (\verb+MAX_WAYPOINTS+) das
\verb+drive_stack()+ Verhlaten aufzurufen.
