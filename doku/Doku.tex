\documentclass[a4paper,12pt]{scrartcl}[1970/01/01]
\usepackage[ngerman]{babel}
\usepackage[T1]{fontenc}
\usepackage[utf8]{inputenc}
\usepackage{graphicx}
\usepackage{url}
\usepackage{cite}
\usepackage{float}
\usepackage{xcolor}
\usepackage[breaklinks=true]{hyperref}
	\definecolor{mygray}{RGB}{240,240,240}
	\definecolor{mywhite}{RGB}{255,255,255}
	\hypersetup{citebordercolor=mywhite,
				filebordercolor=mywhite,
				linkbordercolor=mywhite,
				menubordercolor=mywhite,
				urlbordercolor=mygray,
				runbordercolor=mywhite}
\usepackage{listings}

\title{Mobile Roboter SS12\\Gruppe x}
\author{Kevin Walter, Gerhard Klostermeier, Andreas Jansche}
\date{\copyright\space\today}

\begin{document}
\maketitle
\newpage

\tableofcontents
\newpage

%%%Abschnitt:
%\section{Foo}
%\subsection{SubFoo}
%\subsection{Another SubFoo}
%%Ende Foo
%\newpage

%%%Bild:
%\begin{figure}[htbp]
%\centering
%\includegraphics{bilder/foo}
%\caption{This is a foo}
%\end{figure}

\section{Einleitung}
\subsection{Inhalte und Ziele}
Im Rahmen der Vorlesung "'Mobile Roboter"' der Hochschule Aalen wurden die Aufgaben aus dem Vorlesungsskript sowie eine größere, selbst ausgewählte Aufgabe bearbeitet.
Genutzt wurde Eclipse zum Entwickeln der Programme sowie avrdude zum Flashen des Microcontrollers des ct-Bot.

\subsection{Der ct-Bot}
"`Der c't-Bot ist ein Projekt der Fachzeitschrift c't aus dem Verlag Heinz Heise.
Er soll möglichst vielen Lesern den Zugang zu dem spannenden Thema Robotik eröffnen.
Daher besteht das Projekt aus zwei Teilen: Dem eigentlichen Roboter c't-Bot und dem passenden Simulator c't-Sim.

Den c't-Bot gibt es nur als Bausatz. In der Grundversion besitzt er zwei Räder,
hat eine runde Grundfläche vom Durchmesser einer CD und ist mit einer ganzen Reihe von Sensoren
bestückt. Seine Intelligenz sitzt in einem Mikrocontroller, der in C programmiert wird. Mechanik
und Intelligenz sind ein Tradeoff zwischen Preis, Eleganz und Stabilität. Ganz bewusst kommen keine
SMD-Bauteile zum Einsatz, damit auch unerfahrene Löter eine Chance haben,
sich ihren Spielgefährten aufzubauen.

Der Simulator c't-Sim ist in Java geschrieben und macht ausführlichen Gebrauch von der
3D-Bibliothek Java3D. Er läuft derzeit unter Windows und Linux. Damit man den ganzen Steuer-Code
für den Roboter nur einmal entwicklen muss, ist dieser in C geschrieben. Er läst sich für PC und Mikrocontroller
übersetzen. Auf dem PC nimmt er per TCP/IP Kontakt zum Simulator auf. Dieser versorgt den C-Teil dann mit Sensorwerten.
Auf dem Mikrocontroller liest er dann die echten Sensoren aus. Der Simulator funktioniert auch
ohne den Roboter und der Roboter auch ohne den Simulator.

Das ganze Projekt lebt vom Mitmachen. Der von c't vorgestellte Quelltext liefert zwar ein recht
vollständiges Framework für den Roboter und den Simulator, die Intelligenz des Roboters zu
implementieren bleibt aber den Lesern überlassen. Dennoch fließen pfiffge Patches immer wieder in
den offiziellen Code ein. Der gesamte Code steht unter der GPL."'
\footnote{Quelle: Benjamin Benz, \url{http://wiki.ctbot.de/index.php?title=Hauptseite&oldid=3655}}




%Ende Einleitung
\newpage

\section{Installation und Inbetriebnahme}
\subsection{Installation}
Für die Entwicklung wurden ein aktuelles Ubuntu Linux sowie ein Arch Linux verwendet. In diesem Abschnitt ist beschrieben was für vorbereitende Schritte auf dem System durchgeführt werden müssen um entwickeln und den ct-Bot flashen zu können.
(Die nachfolgenden Anweisungen gelten für Ubuntu.)


\subsubsection{System vorbereiten}
Zunächst müssen einige Softwarepakete nachinstalliert werden:
\begin{itemize}
\item eclipse-cdt\\
Die Eclipse-Variante zur C/C++-Entwicklung. (Der ct-Bot wird in C programmiert.)
\item binutils-avr, gcc-avr, avr-libc\\
Werden zum Kompilieren für den Microcontrollers des ct-Bot benötigt. (Cross-Compiler)
\item avrdude\\
Wird zum Flashen des Microcontroller des ct-Bot benötigt.
\item subversion\\
Wird zum Holen des aktuellen Quellcodes für den ct-Bot aus dem heise-Repository benötigt.
\end{itemize}
Der konkrete Befehl um die Pakete unter Ubuntu zu installieren sieht folgendermaßen aus:
\begin{lstlisting}
	sudo apt-get install eclipse-cdt binutils-avr gcc-avr \
	avr-libc avrdude subversion
\end{lstlisting}


\subsubsection{Quellcode holen}
Für den Quellcode erstellen wir zunächst ein Verzeichnis \textit{ctbot} und wechseln hinein:
\begin{lstlisting}
	mkdir ctbot && cd ctbot
\end{lstlisting}
Nun holen wir uns den aktuellen stable-Code vom heise-Repository:
\begin{lstlisting}
	svn checkout https://www.heise.de:444/svn/ctbot/stable
\end{lstlisting}
Der aktuelle Quellcode des ct-Bot befindet sich nun also unter \textit{\~{}/ctbot/stable} und muss im nächsten Schritt nur noch in Eclipse eingebunden werden.


\subsubsection{Eclipse einrichten}
Zunächst müssen wir den Quellcode in Eclipse einbinden:
\begin{itemize}
\item Dazu wählen wir zunächst im Menü \textit{File} den Unterpunkt \textit{Import}.
\item Dort wählen wir \textit{General -> Existing Projects into Workspace} und bestätigen mit \textit{Next >}.
\item Unter der Auswahl \textit{Select root directory} geben wir entweder direkt das Quellcodeverzeichnis an (\textit{\~{}/ctbot/stable/ct-Bot}) oder wählen das Verzeichnis über \textit{Browse...}.
\end{itemize}
Nun ist der Quellcode als Projekt in Eclipse eingebunden. Um für den ct-Bot zu kompilieren muss jedoch noch die Build-Configuration angepasst werden:
\begin{itemize}
\item Auf der linken Seite wählen wir zunächst das Projekt \textit{ct-Bot} aus.
\item Im Menü \textit{Project} wählen wir \textit{Properties} und dort \textit{C/C++-Build}.
\item Dort gehen wir auf \textit{Manage Configurations...} und wählen die Konfiguration \textit{Debug-MCU-m32}. Mit dieser Konfiguration wird der zuvor installierte Cross-Compiler genutzt um eine hex-Datei zum Flashen des ct-Bot zu erstellen.
\item Wir bestätigen die Auswahl der Konfiguration mit \textit{Set active} und verlassen die Einstellungen mit \textit{Ok}, \textit{Apply} und nochmals \textit{Ok}.
\end{itemize}
Nachdem nun auch die passende Konfiguration gewählt wurde lässt sich das Projekt nun auch kompilieren. Jedoch kommt es zu einigen Warnmeldungen.
Um diese Warnungen loszuwerden öffnen wir die Datei \textit{ct-Bot.h} (\textit{include/ct-Bot.h}) und suchen die Zeile
\begin{lstlisting}
//#define SPEED_CONTROL_AVAILABLE
\end{lstlisting}
und entfernen die Kommentarzeichen \textit{//}.
Nun lässt sich das Projekt ohne Warnungen kompilieren und die  eigentliche Entwicklung kann beginnen.

\subsection{Inbetriebnahme}
\subsubsection{AVR ISP mkII und avrdude}
Der \textit{AVR ISP mkII} ist der Programmer, mit dem wir den ct-Bot flashen können. Als Tool dazu verwenden wir \textit{avrdude}, das den \textit{AVR ISP mkII} ansprechen kann.

Nach dem Einstecken des Programmers können wir mit dem Befehl \textit{lsusb} prüfen, ob er korrekt vom System erkannt wird. Die Ausgabe sollte folgenden Text enthalten:
\begin{lstlisting}
Atmel Corp. AVR ISP mkII
\end{lstlisting}

Um unsere hex-Datei nun auf den ct-Bot zu bekommen, müssen wir \textit{avrdude} mit entsprechenden Parametern aufrufen:
\begin{itemize}
\item \textit{-c avrispmkII} legt den \textit{AVR ISP mkII} als Programmer fest.
\item \textit{-P usb} gibt USB als connection port an.
\item \textit{-p m32} legt m32 (ATmega32) als AVR device fest.
\item \textit{-U flash:w:<pfad>:i} legt die Aktion fest:
	\begin{itemize}
	\item \textit{flash} gibt an, dass geflasht werden soll.
	\item \textit{w} (write) gibt an, dass geschrieben werden soll. (Von der Datei in den Flash-Speicher, \textit{r} (read) würde bedeuten vom Flash in die Datei.)
	\item \textit{<pfad>} gibt den Pfad zu der Datei an.
	\item \textit{i} gibt (optional) das Format der Datei an. (Hier \textit{i} für \textit{Intel Hex}.)
	\end{itemize}
\end{itemize}
Zu beachten ist, dass \textit{avrdude} als root oder über \textit{sudo} aufgerufen werden muss.
Der konkrete Aufruf würde also folgendermaßen aussehen:
\begin{lstlisting}
sudo avrdude -c avrispmkII -P usb -p m32 -U \
flash:w:"~/ctbot/stable/ct-Bot/Debug-MCU-m32/ct-Bot.hex":i
\end{lstlisting}

\newpage

\section{Aufgabenberichte}
\subsection{Motte / Kakalake}
% Motte.txt enthält subsubsections für verschiedene 
% Lösungsansätze.
\subsection{Motte/Kakerlake}
\label{motte_kakerlake}

\subsubsection{Ansatz 1: Verhaltensframework}
Dieser Ansatz versucht die Aufgabe zu lösen in dem das Verhaltensframework
genutzt wird, dass mit dem ct-Bot Reopsitory ausgelierfert wird. Die Nutzung des 
Frameworks hat einige Vorteile:
\begin{itemize}
	\item Eigener Code leich einzusortieren und stört kein vorhandenen Code.
	\item Eigene Verhalten können simpel auf andere Verhalten zugreifen.
	\item Durch die Priorisierung der können mehrere Verhalten zusammen Arbeiten.
		(z.B. Bot soll eine Distnaz von 30cm fahren. Wenn er dabei aber an eine
		Tischkannte kommt wird abgebrochen damit der Bot nicht vom Tisch fällt.)
	\item Der ausgelieferte Verhlatenssatz bietet eine vielzahl von nützlichen
		Verhalten.
\end{itemize}
Nachteil ist leider die etwas komplexe Einarbeitung, da der Einstieg nicht sehr gut
dokumentiert ist. \\

In dieser Lösung wurde die Aufgabe einer Lichtquelle zu folgen (Motte), in drei
Unteraufgabenunterteilt:Linksdrehen, Rechtsdrehen, geradeaus Fahren.
Welche Funktion ausgeführt wird, entscheidet ein Vergelich der beiden Fotowiderstände
(LDR - \textit{Light Dependent Resistor}). In diesen Vergleich fliesen zwei Konstanten ein:
\verb+LDR_CORRECT+ und \verb+TOLERANCE+. Mit \verb+LDR_CORRECT+ lassen sich
(Hardwarebedingte) Unterschiede zwischen den beiden Sensoren ausgleichen. Liefert
der linke Widerstand Beispielsweise bei gleichmäßiger Bestrahlung immer einen
Wert der um 20 höher ist wie der des Rechten, so kann mit \verb+LDR_CORRECT=20+ dieser
Fehler ausgeglichen werden. Die zweite Konstante ist für das geradeaus Fahren wichtig.
Logisch wäre die Lichtquelle direkt vor dem Bot, wenn beide Sensoren den gleichen
Wert liefern. In der Praxis wird das aber selten der Fall sein. Deswegen kann mit
\verb+TOLERANCE+ angegeben werden, wie viel mehr Licht auf den linken bzw. den rechten
Widerstand strahlen darf, ohne das es als links bzw. rechts Fahren interpretiert wird.
Bei einerm \verb+TOLERANCE+ Wert von 15 wird Beispielsweise trozdem geradeaus gefahren,
wenn der linke Sensor einen um 10 höheren Wert wie der rechte aufweist. Jetzt kann
der Roboter grade auf die Lichtquelle zu fahren, selbst wenn diese nicht zu 100\%
vor ihm liegt. \\

Das Verhalten der Kakalake (vor einer Lichtquelle fliehen) ist exakt gleich dem der 
Motte implementiert, nur das rückwärts gefahren wird. So wird die Lichtquelle immer
von der Roboter Vorderseite fixiert und dann davon weggefahren. \\

Die modifizerte Version der Motte, die sich den gefahrenen Weg merken und diesen 
wieder zurückfahren können soll, wurde über vorhandene Verhalten realisiert.
Dazu wurde zuerst das Verhalten \verb+BEHAVIOUR_DRIVE_STACK_AVAILABLE+ zugeschalten \\
(in \verb+ct-Bot/include/bot-logic/available\_behaviours.h+ und anschlißend
auf aktiv gesetzt (\verb+bot_save_waypos_behaviour+ in
\verb+ct-Bot/bot-logic/bot-logic.c+). Dieses Verhalten zeichnet im Folgenden
alle Informationen auf, die für das wiederfinden der relevanten Positonen nötig sind.
Nach einer über die Konstante \verb+MAX_WAYPOINTS+ festgelegten Anzahl Wegpunkten,
wird das \verb+drive_stack()+ Verhlaten aufgerufen das die Punkte wieder anfährt.
Wenn der Roboter an der Ausgangsposition angekommen ist, beginnt er wieder von vorne mit
der Wegaufzeichnung und der Lichtquellensuche. \\

Code einfügen / manipuliernen:
\begin{itemize}
    \item \verb+behaviour_follow_light.c+  im Verzeichnis 
        \verb+ct-Bot/bot-logic/+ einsortieren.
    \item \verb+behaviour_follow_light.h+ in Verzeichnis
        \verb+ct-Bot/include/bot-logic/+ einsortieren.
    \item Des Weiteren sind Änderungen in den Dateinen \verb+ct-Bot/ct-Bot.h+, \\
        \verb+ct-Bot/bot-logic/bot-logic.c+,
        \verb+ct-Bot/ui/available_screens.h+ und \\
        \verb+ct-Bot/include/bot-logic/available_behaviours.h+ zu nötig.
        Wie genau die Änderungen sind ist der \verb+follow-light-diff.txt+ zu entnehmen.
\end{itemize}

\subsubsection{Allgemeiner Ansatz}

Die Aufgabe ist an sich ziemlich simpel da man nur die beiden sensor werte vergleichen muss und angemessen darauf reagieren.
Um die Sensorwerte einfach zu vergleichen wurde hierzu einfach die differenz der beiden werte genommen.
Anhand der Differenz kann mann dann einfach überprüfen ob das Licht mehr von Rechts oder von Links kommmt, oder ob das Licht gleichmäßig verteilt auftrifft.
Ist diese Unterscheidung der Position getan muss noch im Fall von gleichmäßigem auftreffen des Lichts überprüft werden ob es gleichmäßig hell oder gleichmäßig dunkel ist.



\subsubsection{Ansatz 3: by Andy}
...

%\subsection{Line folgen}
%\input{Line-folgen.tex}
%\subsection{Acht (8) fahren}
%\subsection{8 Fahren}
\label{8-fahren}

\subsubsection{Allgemeiner Ansatz}

\verb+void acht_nonbehav()+ \\

Für diese Aufgabe wurde ein relativ einfaches Prinzip gewählt: Es gibt zwei Schleifen nacheinander, welche den Bot zunächst eine Rechtskurve und anschließend eine Linkskurve fahren lassen. Ergeben beide Kurven einen Kreis, dann fährt der Bot eine "'Acht"'. Da die Motorleistung von Bot zu Bot, von rechtem Motor zu linkem Motor oder einfach je nach Batteriestand anders sind ging das Konzept aber so nicht auf.
Um diese Verschiedenheiten auszugleichen wurden die Variablen \verb+static int distr+ und \verb+ static int distl+ eingeführt, um zur Laufzeit Änderungen an der Kurvenfahrt vorzunehmen. Diese Variablen bestimmen wie lange der Bot die Links- (\verb+distl+) bzw. Rechtskurve (\verb+distr+) fährt. So kann die Kurvenfahrt für jede Richtung einzeln justiert werden. 
Hierzu wird in jedem Schleifendurchlauf mit Hilfe von \verb+ir_read()+ der Wert vom Infrarotsensor ausgelesen und in einer switch-case-Anweisung überprüft. Je nach Sensorwert wird dann \verb+distr+ oder \verb+distr+ inkrementiert bzw. dekrementiert. Für die Fernbedienung \textit{TOTAL control} aus dem Labor ergibt sich dafür folgende Steuerung:
\begin{itemize}
	\item \textbf{Pfeil hoch}   : \verb_distr_  inkrementieren
	\item \textbf{Pfeil runter} : \verb_distr_  dekrementieren
	\item \textbf{Pfeil rechts} : \verb_distl_  inkrementieren
	\item \textbf{Pfeil links}  : \verb_distl_  dekrementieren
\end{itemize}

\noindent Die aktuellen Werte von \verb+distr+ und \verb+distl+ werden auf dem Display ausgegeben.\\

\textit{(Anmerkung: Für jede Kurve gibt es je zwei ineinander geschachtelte Schleifen, die beide jeweils von }\verb+0+\textit{ bis }\verb+distr+/\verb+distl+\textit{ laufen. Warum nicht nur eine Schleife verwendet wurde hat einen einfachen Grund: Erhöht man den Kurvenradius durch anpassen der Motorgeschwindigkeiten, muss }\verb+distr+/\verb+distl+\textit{ ebenfalls angepasst werden und erreicht schnell Werte, die ein Integer nicht mehr fassen kann, was zu einem Integer-Überlauf und somit Fehlverhalten führt.)}



\section{Zeitplan}
% Zeitaufwand aller Projekte (aufgelöst und verrechnet)
% Zeitaufwand aller Projekte (aufgelöst und verrechnet) als Tabelle

\begin{center}
	\begin{tabular}{  c | c | c | c |}
		\cline{2-4}
		& \multicolumn{3}{|c|}{Zeit}  \\ 
		\hline
		\multicolumn{1}{|c|}{Aufgabe} & Lösungsfindung & Implementierung & Tweaking \\
		\hline
		\hline
		\multicolumn{1}{|c|}{\nameref{benutzung} \ref{benutzung}}  &  &  & \\ 
		\hline
		\multicolumn{1}{|c|}{\nameref{motte_kakerlake} \ref{motte_kakerlake}}  &  &  & \\ 
		\hline
		\multicolumn{1}{|c|}{\nameref{linie_folgen} \ref{linie_folgen}} &  &  & \\ 
		\hline
		\multicolumn{1}{|c|}{\nameref{8-fahren} \ref{8-fahren}} &  &  &\\ 
		\hline
		\multicolumn{1}{|c|}{\nameref{fernsteuerung} \ref{fernsteuerung}} &  &  & \\ 
		\hline
		\multicolumn{1}{|c|}{\nameref{weg-aufzeichnen} \ref{weg-aufzeichnen}} &  &  & \\ 
		\hline
  \end{tabular}
\end{center}



%\section{Quellen}
%\bibliography{doku}{}
%\bibliographystyle{alpha}

\end{document}

