\subsection{Inhalte und Ziele}
Im Rahmen der Vorlesung "'Mobile Roboter"' der Hochschule Aalen wurden die Aufgaben aus dem Vorlesungsskript sowie eine größere, selbst ausgewählte Aufgabe bearbeitet.
Genutzt wurde Eclipse zum Entwickeln der Programme sowie avrdude zum Flashen des Microcontrollers des ct-Bot.

\subsection{Der ct-Bot}
"`Der c't-Bot ist ein Projekt der Fachzeitschrift c't aus dem Verlag Heinz Heise.
Er soll möglichst vielen Lesern den Zugang zu dem spannenden Thema Robotik eröffnen.
Daher besteht das Projekt aus zwei Teilen: Dem eigentlichen Roboter c't-Bot und dem passenden Simulator c't-Sim.

Den c't-Bot gibt es nur als Bausatz. In der Grundversion besitzt er zwei Räder,
hat eine runde Grundfläche vom Durchmesser einer CD und ist mit einer ganzen Reihe von Sensoren
bestückt. Seine Intelligenz sitzt in einem Mikrocontroller, der in C programmiert wird. Mechanik
und Intelligenz sind ein Tradeoff zwischen Preis, Eleganz und Stabilität. Ganz bewusst kommen keine
SMD-Bauteile zum Einsatz, damit auch unerfahrene Löter eine Chance haben,
sich ihren Spielgefährten aufzubauen.

Der Simulator c't-Sim ist in Java geschrieben und macht ausführlichen Gebrauch von der
3D-Bibliothek Java3D. Er läuft derzeit unter Windows und Linux. Damit man den ganzen Steuer-Code
für den Roboter nur einmal entwicklen muss, ist dieser in C geschrieben. Er läst sich für PC und Mikrocontroller
übersetzen. Auf dem PC nimmt er per TCP/IP Kontakt zum Simulator auf. Dieser versorgt den C-Teil dann mit Sensorwerten.
Auf dem Mikrocontroller liest er dann die echten Sensoren aus. Der Simulator funktioniert auch
ohne den Roboter und der Roboter auch ohne den Simulator.

Das ganze Projekt lebt vom Mitmachen. Der von c't vorgestellte Quelltext liefert zwar ein recht
vollständiges Framework für den Roboter und den Simulator, die Intelligenz des Roboters zu
implementieren bleibt aber den Lesern überlassen. Dennoch fließen pfiffge Patches immer wieder in
den offiziellen Code ein. Der gesamte Code steht unter der GPL."'
\footnote{Quelle: Benjamin Benz, \url{http://wiki.ctbot.de/index.php?title=Hauptseite&oldid=3655}}



