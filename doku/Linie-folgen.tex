\subsection{Linie folgen}

\subsubsection{Ansatz 1: Verhaltensframework}

\subsubsection{Ansatz 2: by Kevin}

Allgemeines Vorgehen\\
Ähnlich wie bei der Motte/Karkalake ist hier nur wieder entscheident die Sensorwerte zu verlgleichen und entsprechend zu reagieren. Auch hier wurde über die Differenz der Sensorwerte zuerst geprüft ob man mit einem der beiden Sensoren von der Linie runter ist, um entsprechend gegenzulenken. Wenn sich beide Sensorwerte kaum unterscheiden muss noch die unterscheidung getroffen werden ob man ganz von der Linie Runter ist oder ob man noch ganz auf der Linie ist.
\\
Da die Testlinie einen seltsamen Verlauf hat (90 Grad abknickend mit ungleichmäßiger Linienbreite) funktioniert der oben beschribene Algorithmus nicht an diesen Stellen. Mit etwas Glück gelingt es ohne Sonderbehandlung aber es ist bisher nicht verlässlich reproduzierbar.


\subsubsection{Ansatz 3: by Andy}
...
