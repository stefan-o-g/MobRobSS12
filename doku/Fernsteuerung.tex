\subsection{IR-Fernsteuerung}
\label{fernsteuerung}

\subsubsection{Allgemeiner Ansatz}

Um die Fernbedienung benutzen zu können muss eigentlich nur der Sensorwert mit \verb|ir_read()| gelesen werden. Beim Lesen wird der Wert gelöscht, so dass man den Code nicht zwei mal liest. Die Fernbedienung setzt bei mehrmaligem drücken der selben Taste das 11. Bit abwechselnd auf 1 und 0. Man muss also die Bits, die zum Befehl gehören und die die Zusatzinformationen beinhalten, trennen. Da das Layout des Codes nirgends dokumentiert und der Rest der Bits stabil zu sein scheint, wurde nach erhalten des Codes einfach das 11. Bit mit \verb|code & ~(1<<11)| auf 0 gesetzt. Somit unterscheiden sich die Codes bei mehrmaligem Drücken nicht mehr. Die verwendete Fernbedienung liefert zudem unterschiedliche Codes für unterschiedliche Geräte (Multifunktionsfernbedienung). Für die Aufgaben wurden die TV Codes der Fernbedienung verwendet.

\textit{(Hinweis: In der Aufgabe "'8 Fahren"' (siehe \ref{8-fahren}) wurde ebenfalls die IR-Fernbedienung genutzt.)}
