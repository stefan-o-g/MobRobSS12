\subsection{Motte/Kakerlake}
\label{motte_kakerlake}

\subsubsection{Ansatz 1: Verhaltensframework}

Dieser Ansatz löst die Aufgabe in dem das Verhaltensframework
genutzt wird, dass mit dem ct-Bot Reopsitory ausgelierfert wird.
Diese Lösung teilte die Aufgabe einer Lichtquelle zu folgen (Motte), in drei
Unteraufgaben:Linksdrehen, Rechtsdrehen, geradeaus Fahren.
Welche Funktion ausgeführt wird, entscheidet ein Vergelich der beiden Fotowiderstände
(LDR - \textit{Light Dependent Resistor}). In diesen Vergleich fliesen zwei Konstanten ein:
\verb+LDR_CORRECT+ und \verb+TOLERANCE+. Mit \verb+LDR_CORRECT+ lassen sich
(Hardwarebedingte) Unterschiede zwischen den beiden Sensoren ausgleichen. Liefert
der linke Widerstand Beispielsweise bei gleichmäßiger Bestrahlung immer einen
Wert der um 20 höher ist wie der des Rechten, so kann mit \verb+LDR_CORRECT=20+ dieser
Fehler ausgeglichen werden. Die zweite Konstante ist für das geradeaus Fahren wichtig.
Logisch wäre die Lichtquelle direkt vor dem Bot, wenn beide Sensoren den gleichen
Wert liefern. In der Praxis wird das aber selten der Fall sein. Deswegen kann mit
\verb+TOLERANCE+ angegeben werden, wie viel mehr Licht auf den linken bzw. den rechten
Widerstand strahlen darf, ohne das es als links bzw. rechts Fahren interpretiert wird.
Bei einerm \verb+TOLERANCE+ Wert von 15 wird Beispielsweise trozdem geradeaus gefahren,
wenn der linke Sensor einen um 10 höheren Wert wie der rechte aufweist. Jetzt kann
der Roboter grade auf die Lichtquelle zu fahren, selbst wenn diese nicht zu 100\%
vor ihm liegt. \\

Das Verhalten der Kakalake (vor einer Lichtquelle fliehen) ist exakt gleich dem der 
Motte implementiert, nur das rückwärts gefahren wird. So wird die Lichtquelle immer
von der Roboter Vorderseite fixiert und dann davon weggefahren. \\

Die modifizerte Version der Motte, die sich den gefahrenen Weg merken und diesen 
wieder zurückfahren können soll, wurde über vorhandene Verhalten realisiert.
Dazu wurde zuerst das Verhalten \verb+BEHAVIOUR_DRIVE_STACK_AVAILABLE+ zugeschalten \\
(in \verb+ct-Bot/include/bot-logic/available\_behaviours.h+ und anschlißend
auf aktiv gesetzt (\verb+bot_save_waypos_behaviour+ in
\verb+ct-Bot/bot-logic/bot-logic.c+). Dieses Verhalten zeichnet im Folgenden
alle Informationen auf, die für das wiederfinden der relevanten Positonen nötig sind.
Nach einer über die Konstante \verb+MAX_WAYPOINTS+ festgelegten Anzahl Wegpunkten,
wird das \verb+drive_stack()+ Verhlaten aufgerufen das die Punkte wieder anfährt.
Wenn der Roboter an der Ausgangsposition angekommen ist, beginnt er wieder von vorne mit
der Wegaufzeichnung und der Lichtquellensuche. \\

Code einfügen / manipuliernen:
\begin{itemize}
    \item \verb+behaviour_follow_light.c+  im Verzeichnis 
        \verb+ct-Bot/bot-logic/+ einsortieren.
    \item \verb+behaviour_follow_light.h+ in Verzeichnis
        \verb+ct-Bot/include/bot-logic/+ einsortieren.
    \item Des Weiteren sind Änderungen in den Dateinen \verb+ct-Bot/ct-Bot.h+, \\
        \verb+ct-Bot/bot-logic/bot-logic.c+,
        \verb+ct-Bot/ui/available_screens.h+ und \\
        \verb+ct-Bot/include/bot-logic/available_behaviours.h+ zu nötig.
        Wie genau die Änderungen sind ist der \verb+follow-light-diff.txt+ zu entnehmen.
\end{itemize}

\subsubsection{Allgemeiner Ansatz}

Die Aufgabe ist an sich ziemlich simpel da man nur die beiden sensor werte vergleichen muss und angemessen darauf reagieren.
Um die Sensorwerte einfach zu vergleichen wurde hierzu einfach die differenz der beiden werte genommen.
Anhand der Differenz kann mann dann einfach überprüfen ob das Licht mehr von Rechts oder von Links kommmt, oder ob das Licht gleichmäßig verteilt auftrifft.
Ist diese Unterscheidung der Position getan muss noch im Fall von gleichmäßigem auftreffen des Lichts überprüft werden ob es gleichmäßig hell oder gleichmäßig dunkel ist.



\subsubsection{Ansatz 3: by Andy}
...
