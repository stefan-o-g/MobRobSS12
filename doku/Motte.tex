\subsection{Motte/Kakerlake}

\subsubsection{Ansatz 1: Verhaltensframework}
Dieser Ansatz versucht die Aufgabe zu lösen in dem das Verhaltensframework
genutzt wird, dass mit dem ct-Bot Reopsitory ausgelierfert wird. Die Nutzung des 
Frameworks hat einige Vorteile:
\begin{itemize}
	\item Eigener Code leich einzusortieren und stört kein vorhandenen Code.
	\item Eigene Verhalten können simpel auf andere Verhalten zugreifen.
	\item Durch die Priorisierung der können mehrere Verhalten zusammen Arbeiten.
		(z.B. Bot soll eine Distnaz von 30cm fahren. Wenn er dabei aber an eine
		Tischkannte kommt wird abgebrochen damit der Bot nicht vom Tisch fällt.)
	\item Der ausgelieferte Verhlatenssatz bietet eine vielzahl von nützlichen
		Verhalten.
\end{itemize}
Nachteil ist leider die etwas komplexe Einarbeitung, da der Einstieg nicht sehr gut
dokumentiert ist.

\subsubsection{Ansatz 2: by Kevin}

Allgemeines Vorgehen \\
Die Aufgabe ist an sich ziemlich simpel da man nur die beiden sensor werte vergleichen muss und angemessen darauf reagieren.
Um die Sensorwerte einfach zu vergleichen wurde hierzu einfach die differenz der beiden werte genommen.
Anhand der Differenz kann mann dann einfach überprüfen ob das Licht mehr von Rechts oder von Links kommmt, oder ob das Licht gleichmäßig verteilt auftrifft.
Ist diese Unterscheidung der Position getan muss noch im Fall von gleichmäßigem auftreffen des Lichts überprüft werden ob es gleichmäßig hell oder gleichmäßig dunkel ist.



\subsubsection{Ansatz 3: by Andy}
...
